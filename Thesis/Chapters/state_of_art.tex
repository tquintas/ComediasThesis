\documentclass[../main.tex]{subfiles}
\begin{document}

De modo a limitar o âmbito do referido projeto, é feita uma revisão de literatura suportada em três temáticas: piano preparado, preparações digitais e instalações artísticas baseadas na arte performática.

Falar de um projeto que envolve o conceito de um \textit{Piano Digitalmente Preparado} assume de imediato as definições de piano preparado e de preparações digitais, tanto no ponto de visto conceptual como nos pontos de vista do ato da performance e do ato da composição. Procuro, assim, recolher bibliografia que tanto aborda as orgiens destas técnicas pianísticas como reflete particularidades performáticas e composicionais das obras para piano preparado e piano digitalmente preparado.

Uma vez que o projeto também consiste na elaboração de uma instalação performática eletroacústica, farei também um levantamento bibliográfico sobre os referidos conceitos, sobre a comunicação do artista com o espaço e o público no âmbito da arte performática e sobre a envolvência de meios eletroacústicos numa instalação artística. Esta temática inclui também conceitos de paisagens sonoras, não num ponto de vista analítico e de reconhecimento, mas como um meio para a criação artística, para moldar o espaço da instalação com apenas intervenção sonora.

Como o ser humano está continuamente exposto a perfis sonoros bastante característicos do meio que o rodeia, essa paisagem sonora torna-se um identificador desse meio \cite{schafer1993}. Este reconhecimento evolui intuitivamente em cada pessoa, mesmo que se trate de ambientes sonoros que esta não tenha propriamente vivenciado.

Esta naturalidade de adaptação sonora às paisagens que nos rodeiam aprimora, talvés inconscientemente, a integração de qualquer pessoa num ambiente performativo artificial, através da expressão criativa quer expontânea, quer conduzida por um líder. Aqui, o conceito de música passa pelo reconhecimento, audição, produção e comunicação dos sons \cite{paynter1992}.

\Subsection{Piano Preparado}
Entende-se por preparar um piano dispor materiais de metal, madeira, plástico e borracha, de diversas formas e tamanhos, entre as cordas de um piano, em diferentes locais da corda \cite{anderson2012}. Embora a primeira aparição de um piano preparado na história da música tenha sido numa performance de Erik Satie, duma obra da sua autoria \cite{dianova2008}, o termo foi cunhado pelo compositor americano John Cage (1912-1992), quando pretendia criar efeitos percussivos num piano, para a performance de uma dança de uma personagem Africana de nome \textit{Bacchanale}, colocando parafusos, porcas e tecido fibroso entre as cordas de doze notas \cite{cage1981, pritchett1993, bunger1981}.

Estas preparações produzem sonoridades modificadas que não refletem o que está escrito numa partitura para este instrumento.
Assim, antes do ato de performance, o intérprete precisa de experimentar e entender que preparações emitem certos resultados, e como inseri-los de forma coerente na obra.
Até no ato de composição, a preparação a antecede, pela experimentação de materiais e de locais \cite{cage1958}.

Além de oferecer novas possibilidades de harmonia, timbre e textura, o piano preparado também pode alterar as percepções auditiva e motora do performer.
Segundo Souza (2017), modificar o instrumento modifica também o intérprete, na medida em que, a execução de uma obra para piano num piano preparado, acusticamente ou digitalmente, danifica a relação entre as percepções.
Todavia, no caso das preparações digitais, há muito mais discrepância, pelo que há uma maior necessidade de “re-aprender” o instrumento pelo intérprete \cite{souza2017}.

Em Portugal, o panorama do piano preparado inclui alguns artigos académicos na área da educação \cite{tsanko2020} e performance \cite{miguel2021}, assim como outros projetos que incluem esta técnica, nomeadamente o \textit{Pianoscópio} \cite{vaz2016,cmt2021}. 

\Subsection{\textsl{bitKlavier}}
De modo a realizar tais preparações, é utilizado o software bitKlavier, idealizado e concretizado por Dan Trueman e Michael Mulshine \cite{trueman2019}.
Aqui, os objetos mecânicos colocados entre as cordas são substituídos por algoritmos colocados entre as cordas virtuais.

De forma semelhante às preparações de Cage, vários módulos podem ser atribuidos a conjuntos de notas, ou teclas, que afectam a morfologia do som, a orientação (normal ou invertido), acionam repetições ou desenhos melódico-harmónicos, alteram a velocidade ou delay de feedback loops de sons, definem temperamentos costumizados, entre outros.
Relativamente a este último tópico, o bitKlavier oferece enúmeras configurações de temperamento estático e adaptativo que não seriam possíveis num piano acústico \cite{trueman2019b}.
Mais ainda, as preparações analógicas demoram tanto ou mais tempo quanto as digitais e com bitKlavier é possível guardar patches, módulos e parâmetros e modificá-los muito rapidamente \cite{trueman2019}.

Utilizando o software, e analisando o manual do mesmo \cite{bk2018}, disponibilizo uma lista sucinta dos vários módulos e preparações que integram o \textit{bitKlaiver}:
\begin{description}
    \bkitem{Gallery} Neste software, uma Gallery (ou galeria) é a encapsulação de um projeto, um modelo de preparações. Podemos pensar nas galerias como uma sala onde está o nosso piano preparado, e o som que dele provém é também influenciado pela acústica da sala.
    \bkitem{Piano} Um Piano representa um conjunto de preparações fixas, como se fosse um subprojeto da galeria. Analogamente, uma Gallery com três Pianos corresponde a uma sala que contém três pianos preparados de forma distinta.
    \bkitem{Keymap} A principal função desta preparação é mapear certas teclas (ou notas) a outras preparações, activando-as. Este módulo também oferece opções de modificar a curvatura da \textit{velocity}\footnote{O termo \textit{velocity} refere-se a um número inteiro de 0 a 127, representativo da amplitude sonora da nota, ou em termos musicais, da dinâmica.}, associar clusters de notas a cada tecla, modificar o funcionamento de outras preparações e especificar os \textit{MIDI inputs}\footnote{Um \textit{MIDI inputs} trata-se de um triplo formado por um número inteiro correspondente à nota, um número inteiro correspondente à \textit{velocity}, e um indicador \textit{on/off} para especificar se a nota foi premida ou largada.} do Keymap.
    \bkitem{Direct} Esta preparação emite diretamente o som do instrumento. Tem alguns parâmetros como os ganhos principal, das resonancias e dos martelos, um envelope ADSR\footnote{ADSR: \textit{attack, decay, sustain, release}} e transposições adicionais. É também possível filtrar os \textit{inputs} de acordo com a sua \textit{velocity}.
    \bkitem{Nostalgic} Este módulo regista o tempo que uma nota está premida e inverte o som que teria sido emitido, com a possibilidade de adicionar transposições. Adicionalmente, é possível indicar um multiplicador ao tempo premido, distância em milisegundos do início do som e tempo adicionado ao tempo premido em \textit{release}. Estes dois últimos parâmetros podem também ser regulados por envelopes ADSR. Esta preparação também consegue filtrar os \textit{inputs} por \textit{velocity}, tempo premido e quantidade de notas presentes num cluster harmónico.
    \bkitem{Synchronic} Com esta preparação é possível construir pulsos, constantes ou inconsistentes e repetitivos, melodias, e outros gestos rítmico-melódicos, especificando as transposições das notas, a sua duração, a duração até à próxima nota e a sua dinâmica, assim como um conjunto de envelopes ADSR. A particularidade deste módulo é que cada um destes parâmetros, apesar de cíclico, não necessita obrigatoriamente de ciclios de igual tamanho, havendo desfaseamento dos parâmetros a cada repetição. É possível também nesta preparação introduzir algumas opções como o número de pulsos e que nota do cluster harmónico inicia a preparação, assim como filtrar os \textit{inputs} por \textit{velocity}, tempo premido e quantidade de notas presentes num cluster harmónico. O Synchronic pode também ser emparelhado com o Nostalgic, possibiltando a sincronização dos pulsos de acordo com a duração das notas invertidas.
    \bkitem{Blendronic} Ao invés de processar os \textit{MIDI inputs}, esta preparação é a primeira a processar o som de outras preparações. O funcionamento deste módulo pode ser pensado como um pedal de \textit{delay} aplicado aos sons das preparações, onde é possível controlar quanto \textit{delay} é aplicado, quanto tempo demoram as suas transições, com que frequência o seu valor é alterado e o quanto enfraquece o som em repetições consecutivas. Assim como no Synchronic, cada um destes parâmetros é cíclico e pode ter tamanhos diferentes, resultando em desconstruções por vezes caóticas do som das preparações. Cada um dos módulos que é processado pelo Blendronic tem um parâmetro de controlo de ganho enviado para esta preparação.
    \bkitem{Resonance} Esta preparação pretende imitar a capacidade ressonante do piano acústico, disponibilizando novas oportunidades de ressonâncias por simpatia. Os principais parâmetros são a nota de referência, as notas que vibram por simpatia e os temperamentos e ganhos para cada nota ressonante. Também há outras opções como o número máximo de vibrações por simpatia e um envelope ADSR.
    \bkitem{Tuning} Como o nome indica, esta preparação serve para alterar o temperamento dos \textit{MIDI inputs}, acossiados aos módulos a que Tuning está ligado. Além duma coleção extensa de sistemas de temperamento históricos e da possibilidade de criação de um temperamento costumizado, o \textit{bitKlavier} também oferece alguns sistemas de afinação adaptativa como a fundamental em andamento\cite{code2002} e \textit{spring tuning}\cite{sethares2005}.
    \bkitem{Tempo} Esta preparação representa um metrónomo que controla os pulsos do Synchronic e do Blendronic a que estão conectados. Este metrónomo pode ser estático ou adaptativo, mutando-se conforme várias notas vão sendo tocadas.
    \bkitem{Piano Map} Esta preparação é um simples mapeamento de um Keymap a um piano, permitindo a mudança entre pianos ao premir uma certa nota ou conjunto de notas.
    \bkitem{Modifications} O objetivo deste módulo é de introduzir uma modificação a outra preparação quando uma certa nota for premida. Para maior parte dos parâmetros, a modificação ocorre de imediato, porém em alguns parâmetros é possível definir um valor a ser incrementado, um máximo de vezes que o valor é incrementado e o tempo de transição da modificação, providênciando alterações mais \textit{smooth}. Quando uma modificação está programada para alterar um certo parâmetro, dizemos que este está \textit{dirty}. Estes módulos também têm a opção de alternar as modificações, alterando todos os parâmetros \textit{dirty} para o seu valor original.
    \bkitem{Reset} O Reset é um módulo que apenas desfaz as modificações que afectaram uma certa preparação, convertendo os parâmetros \textit{dirty} para o seu valor original, quando uma determinada nota é premida. 
    \bkitem{Comment} Este módulo não é em si uma preparação, trata-se apenas de uma caixa de texto onde pode ser introduzido um comentário.
    \bkitem{Compressor} O Compressor é aplicado a toda a Gallery, e tem o funcionamento normal de um compressor de áudio.
    \bkitem{Equalizer} Semelhante ao Compressor, o Equalizer é um equalizador que afecta toda a Gallery.
\end{description}

Antes do lançamento do sotfware, o criador Dan Trueman compôs um conjunto de estudos com recurso a estas preparações e, devido ao excelente acolhimento da obra, foi lançado o desafio a diversos compositores a escrever para bitKlavier.
Assim, há agora apenas dois volumes de obras publicados com recurso a este freeware \cite{trueman2015a,trueman2015b}, juntamente com algumas obras individuais, compostas por Dan Trueman.

\Subsection{Instalações Performativas}

A linha que separa uma instalação artística de uma instalação de arte está muito desfocada \cite{bishop2005}. De acordo com Reiss (1999), em ambas existem interações triangulares entre espectador, obra e espaço, porém, numa instalação artística o espectador é uma parte integrante da instalação \cite{reiss1999}. A autora explora os diveros universos de uma instalação artística, abordando todas as nuances históricas e sócio-políticas que conduziram a expressões artísticas distintas acerca do que significa uma instalação artística.

A partir de 1956, o artista americano Allan Kaprow começa a fazer progressos na evolução artística e performática de uma instalação artística, com a criação de Montagens, Ambientes e Acontecimentos\footnote{originalmente \textit{Assemblages}, \textit{Environments} e \textit{Happenings}}\cite{kaprow1956}. De acordo com o autor, o primeiro consiste em instalações formadas por materiais nos arredores, o segundo acrescenta a interação do público com tais materiais e o último retira a repetitividade das interações, fundindo o quotidiano com a instalação, retirando qualquer conceito artístico \cite{kaprow1956}, descrito como uma performance não-matricial \cite{kirby1965}. Estes \textit{happenings} levaram à criação da arte performática.

Segundo a gramática do artista inglês Howell acerca da arte performática, esta constitui três conceitos principais: Quietude, Repetição e Inconsistência\footnote{originalmente \textit{Stillness}, \textit{Repetition} e \textit{Inconsistency}}\cite{howell1999}. O primeiro engloba a paragem abrupta, o estado meditativo e transição para o movimento; o segundo é o aspecto principal, associado à obcessão; o terceiro é a transferência de uso da repetição para continuamente estimular o gesto artístico. Enquanto disciplina totalmente distinta da do teatro, o autor aborda esta atividade de um ponto de vista tanto artístico como psicoanalítico \cite{howell1999}. 

Com base nestes artigos, posso então definir a questão performática do meu projeto. Consistindo numa instalação de arte, baseada num \textit{Environment}, a interpretação da obra engloba um artista no centro que comunica com o espaço e com o público, dando ênfase à arte performática, pela grmática definida.

\end{document}