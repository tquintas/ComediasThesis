\documentclass[../main.tex]{subfiles}
\begin{document}

\subsection*{Cenários}
Esquematizar as preparações um piano, seja mecanicamente seja digitalmente, engloba a gestão e distribuição ponderada dos sons pelo instrumento, uma vez que a cada nota corresponde apenas um timbre e este não se altera ao longo da obra.
Embora no piano digital estas preparações possam sofrer modificadores ao longo do tempo, completar o esquema de sons, alterações e lançamentos de samples assemelha-se a um jogo de sudoku.
Como o número mínimo de pistas que um jogo de sudoku necessita para ter solução única é 17 \cite{mcguire2012}, elaborei também 17 cenários, que serão retratados em cada peça, como se fossem as pistas que tornam única (para o público) a solução da instalação.

Cada cenário é unicamente descrito por um local e uma emoção. Foram baseados em referências pessoais, garantindo a diversidade sensorial

\end{document}