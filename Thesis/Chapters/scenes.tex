\documentclass[../main.tex]{subfiles}
\begin{document}

%escrever algo

\Subsection{Cenários}
Esquematizar as preparações de um piano, seja mecanicamente seja digitalmente, engloba a gestão e distribuição ponderada dos sons pelo instrumento, uma vez que a cada nota corresponde apenas um timbre e este não se altera ao longo da obra.
Embora no piano digital estas preparações possam sofrer modificadores ao longo do tempo, completar o esquema de sons, alterações e lançamentos de samples implica uma análise cuidada da partitura, horizontal e verticalmente, uma vez que, no caso das preparações digitais, a experimentação não antecede a composição, mas acompanha-a. Assim, esta atribuição de preparações assemelha-se a um jogo de sudoku, como se as notas e preparações fossem os números a serem colocados nos pentagramas da partitura.
Como o número mínimo de pistas que um jogo de sudoku necessita para ter solução única é desassete \cite{mcguire2012}, elaborei também desassete cenários, que serão retratados em cada peça, como se fossem as pistas que tornam única (para o público) a solução da instalação.

Cada cenário é unicamente descrito por um local e uma emoção. Foram baseados em referências pessoais, garantindo a diversidade sensorial e de oportunidade musical, disponíveis através das preparações digitais e da linguagem improvisada do artísta e do público. Partindo da importância da Quietude como alicerce para a Repetição e Inconsistencia na arte performática \cite{howell1999}, a construção dos cenários baseou-se no uso da Quietude, por meio dos exercícios performáticos do \textit{workshop} proposto por Howell, sendo também usados ocasionalmente outros de outras temáticas abordadas pelo autor, incorporando a atividade inerente do público na performance da instalação, tornando estes cenários em possíveis Ambientes.
%falar da stillness, que cada cenario parte da stillness como alicerce para a repetição e inconsistencia, onde depois entra o conceito de environment

\end{document}