\documentclass[../main.tex]{subfiles}
\begin{document}

\Subsection{Composição}

Conforme referi acima, a elaboração da partitura e a experimentação das preparações aconteceram simultaneamente, em abordagens quer artísticas, com a exploração de timbres e texturas, quer técnicas, para assegurar a estrutura e estabilidade de certos módulos. Por vezes a escrita pianística provia de imitações texturais das preparações, por outras as preparações procuravam dialogar ou fundir com o instrumento.

Tendo sido feito primeiro um manuscrito, a formatação das partituras consiste em 9 pentagramas por sistema, um sistema por página: um pentagrama para cada um dos canais de preparações, e quatro pentagramas para o piano, distribuídos em quatro registos diferentes. Esta distribuição é justificada por questões de praticalidade composicional, na disposição espacial estática dos registos na partitura e no alinhamento vertical entre o piano e as preparações. Este manuscrito torna-se ideal para acompanhar em tempo real o intérprete e as interações das preparações com o piano, porém mostra-se pouco prático para o estudo e leitura da obra.

%figura da partitura

Tendo em consideração uma perpectiva mais amigável no ponto de vista do intérprete, a musicografia digital das partituras envolveu processos de condensação e filtragem, mantendo referências visuais importantes das preparações e comprimindo os quatro pentagramas referentes aos registos do piano para dois pentagramas. Este processo envolveu supressões totais das preparações da partitura, em situações onde as preparações eram estáticas e apenas respondiam aos movimentos do intérprete, reduções das preparações num único pentagrama e supressões parciais, mantendo apenas algumas referências visuais, como ajuda para o intérprete. Ainda assim, alguns andamentos foram digitalmente transcritos na íntegra, ou com redução na parte do piano, quando a praticalidade da partitura não se demonstrava suficientemente benéfica para o intérprete, ou quando as referências visuais eram significativamente densas ao longo da partitura.

%exemplos da condensação (em pares)

\end{document}