\documentclass[../main.tex]{subfiles}
\begin{document}

De modo a limitar o âmbito do referido projeto, é feita uma revisão de literatura suportada em 3 temáticas: piano preparado, preparações digitais e instalações artísticas baseadas na arte performática.

Para o primeiro, recorro a conceitos explorados por John Cage (1967)\cite{cage1961} sobre piano preparado, segundo Simon Peter Anderson (2012)\cite{anderson2012} e, como ponte para o segundo tema, abordo as interações entre o intérprete e um instrumento preparado, exploradas por Jonathan de Souza (2017)\cite{souza2017}.

De seguida, referente à digitalização das preparações ao piano, analiso um artigo publicado por Dan Trueman e Michael Mulshine (2019)\cite{trueman2019} que incidem sobre o propósito e funcionamento do software bitKlavier, um artigo por Dan Trueman, Aatish Bhatia \textsl{et al.} (2019)\cite{trueman2019b} sobre questões de temperamento dentro do software, e as notas de programa de 2 volumes de obras compostas com recurso a tal freeware \cite{trueman2015a,trueman2015b}.

Por último, defino os conceitos de instalação artística e de arte performática, segundo Claire Bishop (2005)\cite{bishop2005}, Julie Reiss (1999)\cite{reiss1999} e Marvin Carlson (1996)\cite{carlson1996}.

\subsection*{Piano Preparado}
Entende-se por preparar um piano dispor materiais de metal, madeira, plástico e borracha, de diversas formas e tamanhos, entre as cordas de um piano, em diferentes locais da corda \cite{anderson2012}.
Estas preparações produzem sonoridades modificadas que não refletem o que está escrito numa partitura para este instrumento.
Assim, antes do ato de performance, o intérprete precisa de experimentar e entender que preparações emitem certos resultados, e como inseri-los de forma coerente na obra.
Até no ato de composição, a preparação a antecede, pela experimentação de materiais e de locais \cite{cage1958}.

No entanto, além de oferecer novas possibilidades de harmonia, timbre e textura, o piano preparado também pode alterar as percepções auditiva e motora do performer.
Segundo Souza (2017), modificar o instrumento modifica também o intérprete, na medida em que, a execução de uma obra para piano num piano preparado, acusticamente ou digitalmente, danifica a relação entre as percepções.
Todavia, no caso das preparações digitais, há muito mais discrepância, pelo que há uma maior necessidade de “re-aprender” o instrumento pelo intérprete \cite{souza2017}.

\subsection*{\textsl{bitKlavier}}
De modo a realizar tais preparações, é utilizado o software bitKlavier, idealizado e concretizado por Dan Trueman e Michael Mulshine \cite{trueman2019}.
Aqui, os objetos mecânicos colocados entre as cordas são substituídos por algoritmos colocados entre as cordas virtuais.

De forma semelhante às preparações de Cage, vários módulos podem ser atribuidos a conjuntos de notas, ou teclas, que afectam a morfologia do som, a orientação (normal ou invertido), acionam repetições ou desenhos melódico-harmónicos, alteram a velocidade ou delay de feedback loops de sons, definem temperamentos costumizados, entre outros.
Relativamente a este último tópico, o bitKlavier oferece enúmeras configurações de temperamento estático e adaptativo que não seriam possíveis num piano acústico \cite{trueman2019b}.
Mais ainda, as preparações analógicas demoram tanto ou mais tempo quanto as digitais e com bitKlavier é possível guardar patches, módulos e parâmetros e modificá-los muito rapidamente \cite{trueman2019}.

Antes do lançamento do sotfware, o criador Dan Trueman compôs um conjunto de estudos com recurso a estas preparações e, devido ao excelente acolhimento da obra, foi lançado o desafio a diversos compositores a escrever para bitKlavier.
Assim, há agora apenas 2 volumes de obras publicados com recurso a este freeware \cite{trueman2015a,trueman2015b}, juntamente com algumas obras individuais, compostas por Dan Trueman.

\subsection*{Instalações Performativas}
De acordo com Bishop (2005), a linha que separa uma instalação artística de uma instalação de arte está muito desfocada \cite{bishop2005}.
Em ambas existem interações triangulares entre espectador, obra e espaço, porém, numa instalação artística o espectador é uma parte integrante da instalação \cite{reiss1999}.
Com base nestes artigos, posso então definir a questão performática do meu projeto.
Consistindo numa instalação de arte, a interpretação da obra engloba um artista no centro que comunica com o espaço e com o público, dando ênfase à arte performática \cite{carlson1996}.

\end{document}